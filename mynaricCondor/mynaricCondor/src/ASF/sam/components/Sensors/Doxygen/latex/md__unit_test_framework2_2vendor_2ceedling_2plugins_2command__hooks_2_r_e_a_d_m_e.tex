\chapter{ceedling-\/command-\/hooks}
\hypertarget{md__unit_test_framework2_2vendor_2ceedling_2plugins_2command__hooks_2_r_e_a_d_m_e}{}\label{md__unit_test_framework2_2vendor_2ceedling_2plugins_2command__hooks_2_r_e_a_d_m_e}\index{ceedling-\/command-\/hooks@{ceedling-\/command-\/hooks}}
Plugin for easily calling command line tools at various points in the build process

Define any of these sections in \+:tools\+: to provide additional hooks to be called on demand\+:


\begin{DoxyCode}{0}
\DoxyCodeLine{:pre\_mock\_generate}
\DoxyCodeLine{:post\_mock\_generate}
\DoxyCodeLine{:pre\_runner\_generate}
\DoxyCodeLine{:post\_runner\_generate}
\DoxyCodeLine{:pre\_compile\_execute}
\DoxyCodeLine{:post\_compile\_execute}
\DoxyCodeLine{:pre\_link\_execute}
\DoxyCodeLine{:post\_link\_execute}
\DoxyCodeLine{:pre\_test\_fixture\_execute}
\DoxyCodeLine{:pre\_test}
\DoxyCodeLine{:post\_test}
\DoxyCodeLine{:pre\_release}
\DoxyCodeLine{:post\_release}
\DoxyCodeLine{:pre\_build}
\DoxyCodeLine{:post\_build}

\end{DoxyCode}


Each of these tools can support an \+:executable string and an \+:arguments list, like so\+:


\begin{DoxyCode}{0}
\DoxyCodeLine{:tools:}
\DoxyCodeLine{\ \ :post\_link\_execute:}
\DoxyCodeLine{\ \ \ \ :executable:\ objcopy.exe}
\DoxyCodeLine{\ \ \ \ :arguments:}
\DoxyCodeLine{\ \ \ \ \ \ -\/\ \$\{1\}\ \#This\ is\ replaced\ with\ the\ executable\ name}
\DoxyCodeLine{\ \ \ \ \ \ -\/\ output.srec}
\DoxyCodeLine{\ \ \ \ \ \ -\/\ -\/-\/strip-\/all}

\end{DoxyCode}


You may also specify an array of executables to be called in a particular place, like so\+:


\begin{DoxyCode}{0}
\DoxyCodeLine{:tools:}
\DoxyCodeLine{\ \ :post\_test:}
\DoxyCodeLine{\ \ \ \ -\/\ \ :executable:\ echo}
\DoxyCodeLine{\ \ \ \ \ \ \ :arguments:\ "{}\$\{1\}\ was\ glorious!"{}}
\DoxyCodeLine{\ \ \ \ -\/\ \ :executable:\ echo}
\DoxyCodeLine{\ \ \ \ \ \ \ :arguments:}
\DoxyCodeLine{\ \ \ \ \ \ \ \ \ -\/\ it\ kinda\ made\ me\ cry\ a\ little.}
\DoxyCodeLine{\ \ \ \ \ \ \ \ \ -\/\ you?}

\end{DoxyCode}


Please note that it varies which arguments are being parsed down to the hooks. For now see {\ttfamily command\+\_\+hooks.\+rb} to figure out which suits you best. Happy Tweaking! 